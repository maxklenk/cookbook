\begin{recipe}[]{Saltimbocca a la romana} % MUTTI Feb 14
	\timerecipe[Minuten]{ca. } %mit [EINHEIT]
	\personcount[Personen]{4} % mit[ART]
	\ingredient{8 kleine Kalbsschnitzel} % ggf. \nicefrac{1}{2}
	\ingredient{4 luftgetrockneter Schinken}
	\ingredient{8-16 Salbeiblättchen}
	\ingredient{50ml trockener Weißwein}

\step
Die \textbf{8 Kalbsschnitzel} mit dem Handballen etwas flach drücken.

\step
Jedes Schnitzel mit einem \textbf{halben Schinkenstück} und \textbf{1-2 Salbeiblättchen} belegen und mit Zahnstocher fest stecken. 

\step
Die Schnitzel auf der unbelegten Seite \textbf{salzen} und \textbf{pfeffern}. 

\step
Den Backofen auf 70 °C vorheizen. 

\step
\textbf{2 Eßl. Butter} in einer Pfanne zerlassen. Die Hälfte der Schnitzel mit der Salbeiseite nach unten hineinlegen und bei mittlerer Hitze 2 min braten. Wenden und nochmals 1 min braten, dann auf einem Teller im Ofen warm stellen.

\step
Die restliche Schnitzel wie beschrieben braten und in den Ofen geben.

\step
\textbf{50ml Wein} in die Pfanne gießen und den Bratensatz damit loskochen. Sauce über die Schnitzel gießen rasch servieren. Dazu schmecken Weißbrot oder Nudeln.

%\tippbox{{Tipp:} ...} % Tipp in extra Rahmen
\end{recipe}