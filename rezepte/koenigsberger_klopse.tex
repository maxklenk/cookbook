\begin{recipe}[]{Königsberger Klopse} % MUTTI Nov
	\timerecipe[Minuten]{ca. 60 } %mit [EINHEIT]
	\personcount[Personen]{4} % mit[ART]
	\ingredient{500g Hackfleisch gemischt} % ggf. \nicefrac{1}{2}
	\ingredient{2 Scheiben Weißbrot}
	\ingredient{2 Zwiebel}
	\ingredient{2 Nelken}
	\ingredient{3 EL Petersilie}
	\ingredient{3 EL Sahne}
	\ingredient{1 Lorbeerblatt}
	\ingredient{3 EL Butter}
	\ingredient{2 EL Mehl}
	\ingredient{2 EL Zitronensaft}
	\ingredient{2 EL Kapern}

\step
Aus den \textbf{500g Hackfleisch}, den \textbf{2 Scheiben Weißbrot},
\textbf{einer klein gehackten Zwiebel} und \textbf{3 EL Petersilie} 
einen Hackfleischteig herstellen und Klopse formen.

\step
\textbf{1 Liter Salzwasser} zum Kochen bringen und Temperatur runter schalten. 

\step
Das \textbf{Lorbeerblatt} und die mit den \textbf{2 Nelken} gespickte \textbf{Zwiebel} zugeben.
Die Klopse in das nicht mehr kochende Wasser geben und 15-20 min gar ziehen lassen.

\step
\textbf{2 EL Mehl} in \textbf{2EL zerlassener Butter} anschwitzen, bis es goldgelb ist. Dann den Topf von der Herdplatte nehmen und die Flüssigkeit nach und nach einrühren (Schneebesen).

\step
Die Soße 10 min kochen lassen uns dann \textbf{2 EL Kapern}, \textbf{2 EL Zitronensaft} und \textbf{3 EL Sahne} hinzugeben. Klopse hineigeben und abschmecken.

%\tippbox{{Tipp:} ...} % Tipp in extra Rahmen
\end{recipe}