\begin{recipe}[]{Donauwelle}
	\timerecipe[Minuten]{ca. 40 + 40} %mit [EINHEIT]
	\personcount[Blech]{1} % mit[ART]
	\ingredient{250g Margarine} % ggf. \nicefrac{1}{2}
	\ingredient{400g Zucker}
	\ingredient{350g Mehl}
	\ingredient{6 Eier}
	\ingredient{1 Pk. Backpulver}
	\ingredient{2 Esslöffel Kakaopulver(zum backen)}
	\ingredient{1 Glas Kirschen}
	\ingredient{1 Liter Milch}
	\ingredient{2 Pk. Vanillepuddingpulver}
	\ingredient{250g Butter}
	\ingredient{2 Pk. dunkle Kuvertüre}
	

\step
Backofen auf \textbf{180°C Ober-Unterhitze} vorheizen.

\step
\textbf{250g Margarine}, \textbf{250g Zucker}, \textbf{350g Mehl}, \textbf{1 Päckchen Backpulver} und \textbf{6 Eier} zu einem glatten Teig verrühren.

\step
Die Hälfte des Teiges auf ein mit Backpapier ausgelegtes Blech streichen. Zu der anderen Hälfte \textbf{2 Esslöffel Kakaopulver} mischen und auf den hellen Teig streichen. Das abgetropfte \textbf{Glas Kirschen} gleichmäßig auf dem Teig verteilen.

\step
Bei 180°C, Ober-Unterhitze ca. 30-40 min backen und den Kuchen danach abkühlen lassen.

\step
\textbf{800ml Milch} mit \textbf{250g Butter} und \textbf{150g Zucker} verrühren und zum Kochen bringen. Die mit den restlichen \textbf{200ml Milch} verrührten \textbf{2 Päckchen Puddingpulver} in die kochende Milch rühren und vom Herd nehmen.

\step
Die Creme auf dem Kuchen verteilen und abkühlen lassen.

\step
\textbf{2 Päckchen dunkle Kuvertüre} im Wasserbad schmelzen und auf dem Kuchen verteilen.




%\tippbox{{\bf Tipp:} ...} % Tipp in extra Rahmen
\end{recipe}