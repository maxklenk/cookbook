\begin{recipe}[]{Geschmorte weiße Bohnen} % MUTTI Feb 14
	\timerecipe[Minuten]{ca. } %mit [EINHEIT]
	\personcount{} % mit[ART]
	\ingredient{200g kleine, weiße Bohnen (Cannellini)} % ggf. \nicefrac{1}{2}
	\ingredient{2 Lorbeerblätter}
	\ingredient{1 getrockneter Peperoncino}
	\ingredient{1 Dose Tomaten}
	\ingredient{3 Knoblauchzehen}
	\ingredient{4 Zweige Salbei}


\step
Die \textbf{Bohnen} mit kaltem Wasser bedecken und über Nacht einweichen. 

\step
Am nächsten Tag abgießen und mit frischem Wasser in den Topf füllen.

\step
\textbf{2 Lorbeerblätter} und \textbf{2 Teel. Pfefferkörner} mit leicht angedrücktem \textbf{Peperoncino} dazugeben. 

\step
Die Bohnen zum Kochen bringen und bei schwacher Hitze zugedeckt ca. 1-1\nicefrac{1}{2} Stunden fast weich garen. Am besten zwischendurch mal probieren.

\step
\textbf{3 Koblauchzehen} schälen und kleinhacken, \textbf{Salbeiblätter} in feine Streifen schneiden und beides zusammen in \textbf{Olivenöl} anbraten. Die \textbf{Dose Tomaten} dazugeben und köcheln lassen. 

\step
Die Bohnen abgießen, Lorbeer, Pfeffer und Peperoncino entfernen und zu den Tomaten geben. 
Alles mit \textbf{Salz} und \textbf{Pfeffer} würzen und noch ca. \nicefrac{1}{2} Stunde köcheln, bis die Bohnen weich genug sind.

\tippbox{{Tipp:} Passt als Beilage zu Fleisch mit Nudeln oder Kartoffeln oder Weißbrot.} % Tipp in extra Rahmen
\end{recipe}