\begin{recipe}[]{Frikadellen} %http://www.chefkoch.de/rezepte/819141186405890/Omas-beste-Frikadellen.html
	\timerecipe[Minuten]{ca. ?? } %mit [EINHEIT]
	\personcount{4} % mit[ART]
	\ingredient{500g Hackfleisch (gemischt)} % ggf. \nicefrac{1}{2}
	\ingredient{1 Zwiebel}
	\ingredient{1 Brötchen}
	\ingredient{1 Ei}
	\ingredient{1 TL Senf}
	\ingredient{Semmelbrösel}

\step
\textbf{Ein Brötchen} in Wasser einweichen. Die \textbf{Zwiebel} schälen und in feine Würfel schneiden und kurz anbraten.

\step
Das \textbf{Ei}, die \textbf{Zwiebel}, \textbf{1 TL Salz}, \textbf{1 TL Senf}, \textbf{1 TL Paprikapulver} und \textbf{viel Pfeffer} zum \textbf{Hackfleisch} geben und sehr gut mit den Händen vermengen. 

\step
Die Brötchenmasse sehr gut ausdrücken, zur Hackmasse geben und wieder gut vermengen.

\step
Jetzt gleichmäßige Bällchen formen und auf einer be\textbf{mehl}ten Arbeitsfläche flachdrücken und anschließend in \textbf{Semmelbrösel} wenden. 

\step
Eine Pfanne mit \textbf{Butter} stark erhitzen und die Frikadellen einlegen, kurz auf beiden Seiten scharf anbraten und dann ca. 15 - 20 Minuten (1 - 2mal vorsichtig wenden) auf mittlerer/schwacher Hitze fertig braten.

%\tippbox{{\bf Tipp:} ...} % Tipp in extra Rahmen
\end{recipe}