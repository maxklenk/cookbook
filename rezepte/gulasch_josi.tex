\begin{recipe}[]{Gulasch für Josi und Max} %Max
	\timerecipe[Minuten]{ca. ?? + 120} %mit [EINHEIT]
	\personcount{2} % mit[ART]
	\ingredient{500g Rindergulasch} % ggf. \nicefrac{1}{2}
	\ingredient{2 große Zwiebeln}
	\ingredient{2 Knoblauchzehen}
	\ingredient{2 Gemüsepaprika}
	\ingredient{etwas Tomatenmark}
	\ingredient{1 Becher Sahne}
	\ingredient{(1 Glas Rotwein)}
	
\step
\textbf{500g Rindergulasch} in Öl portionsweise scharf anbraten.

\step
Die in grobe Ringe geschnittenen \textbf{2 große Zwiebeln} und die kleingehackten \textbf{2 Knoblauchzehen} dazugeben und kurz anbraten.

\step
Die \textbf{2 Paprika} in Streifen geschnitten dazugeben und nach kurzem anbraten mit \textbf{Gemüsebrühe} und evt. \textbf{einem Glas Rotwein} aufgießen.

\step
Pfeffer und Salz dazugeben und bei geringer Hitze zugedeckt schmurgeln lassen. Immer mal kontrollieren, ob genug Flüssigkeit da ist.

\step
Nach ca. 1,5-2 Stunden \textbf{einen Becher Sahne} dazugeben und mit Paprikapulver abschmecken. Tabasco macht das Ganze schärfer.


%\tippbox{{Tipp:} ...} % Tipp in extra Rahmen
\end{recipe}