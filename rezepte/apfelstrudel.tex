\begin{recipe}[]{Apfelstrudel} %MUTTI Nov
	\timerecipe[Minuten]{ca. } %mit [EINHEIT]
	\personcount[Blech]{1} % mit[ART]
	\ingredient{250g Mehl} % ggf. \nicefrac{1}{2}
	\ingredient{1 Ei}
	\ingredient{2kg Boskoop}
	\ingredient{1 Packung Rosinen}
	\ingredient{Vanillesoße}

\step
Die \textbf{250g Mehl} in eine Schüssel geben und \textbf{ein Ei}, \textbf{eine Prise Salz}, \textbf{etwas Wasser} und \textbf{ein TL Butter} in die Mitte geben. Alles zügig verkneten. Den Teig unter einem erwärmten Topf ca. \nicefrac{1}{2} Stunde ruhen lassen.

\step
In der Zwischenzeit die \textbf{2kg Boskoop} schälen und in Würfel schneiden.
Die \textbf{Rosinen} in warmem Wasser einweichen.

\step
\textbf{4 EL Semmelbrösel} in einem Topf mit \textbf{3 EL Butter} bräunen und unter die Äpfel mischen.

\step
Den Teig auf einem be\textbf{mehl}ten Küchenhandtuch sehr dünn ausrollen (so dass man die Zeitung durchlesen kann), und die Äpfel mit den Rosinen und dem \textbf{Zimt} und \textbf{Zucker} in der Mitte verteilen. Den Teig zuklappen, die Enden einschlagen und mit Hilfe des Küchenhandtuchs auf ein Backblech mit Backpapier legen.

\step
Mit etwas geschmolzener \textbf{Butter} bestreichen und bei 200°C ca. \nicefrac{1}{2} Stunde backen. Auf die Farbe achten!

\step
Aus dem Ofen nehmen und mit Puderzucker bestreuen und noch warm mit kalter Vanillesoße servieren.


%\tippbox{{Tipp:} ...} % Tipp in extra Rahmen
\end{recipe}